\include{settings} 	% подключение настроек
\begin{document}	     % начало документа
\include{titlepage}   % Титульная страница

\section{Цель работы}
Изучение принципов программирования сокетов протоколов TCP.

\section{Программа работы}
\begin{itemize}
\item разработать простейший клиент и сервер на основе протоколов TCP 
\item разработать прикладной протокол в соответствии с индивидуальным заданием, реализовать протокол в виде клиент-серверного приложения на основе протоколов TCP
\end{itemize}

\section{Ход выполнения работы}

\subsection{Простейшие клиент и сервер}
Простейшие клиент и сервер были выполнены на основе протоколов TCP, а также адаптированы под ОС Windows и Linux. Сервер выполняет функции эхо-сервера, т.е. принимает сообщения от клиентов и посылает копии обратно. Клиент посылает сообщение, после чего завершается.

\subsection{Индивидуальное задание}
Разработать приложение клиент и приложение сервер системы поиска/публикации новостей. Новости сгруппированы по темам. Каждая новость имеет уникальный идентификатор, название и текст новости.

Серверное приложение реализует следующие функции:
\begin{itemize}
\item Прослушивание определенного порта

\itemОбработка запросов по подключению

\item Поддержка одновременной работы нескольких клиентов системы

\item Прием запросов от клиента на передачу списка тем, списка новостей по теме, текста новости, добавление новости по теме

\item Осуществление добавления тем, новостей по темам

\item Передача списков тем, списков новостей и текстов новостей

\item Обработка запроса на отключение клиента

\item Принудительное отключение клиента
\end{itemize}

Клиентское приложение реализует следующие функции:

\begin{itemize}
\item Установление соединения с сервером

\item Передача запросов на передачу списка тем, списка новстей по теме, текста новости, добавление новости по теме

\item Получение результатов от сервера

\item Разрыв соединения

\item Обработка ситуации отключения клиента сервером
\end{itemize}


\subsubsection{Описание протокола}

Для реализации данной системы был разработан текстовый асинхронный протокол. Максимальная длина сообщения 1000 символов. 

Сообщение клиента всегда содержит команду, определяющее тип сообщения. Некоторые типы сообщений содержат также поле опций.

Поле команды имеет размер до 5 байт. Поле опций может (в зависимости от команды) отсутствовать. Опции и команды разделяются пробелом.

Сообщение клиента  содержит команду, определяющее тип сообщения. 
\begin{itemize}
\item Команда для получения списка тем - 1
\item Команда для добавления новости - 2
\item Команда для выхода клиента - 3
\end{itemize}
В любом другом случае, появится сообщение об ошибке.\\

Сообщение сервера содержит команду, определяющее тип сообщения.
\begin{itemize}
\item Команда отключения клиента - kill. Опции - номер клиента.
\item Просмотр подключенных клиентов  - online
\item Закрытие системы - shutdown
\end{itemize}


\subsubsection{Описание структуры приложения}
\paragraph{Сервер:}
\begin{itemize}
\item Создание сокета
\item Создание потока для прослушивания сокета
\item Чтеник команд от сервера
\item Чтеник команд от клиента
\item Получение сообщения от клиентов
\item Анализ сообщения
\item Ответ на сообщение
\end{itemize}

\paragraph{Клиент:}
\begin{itemize}
\item Чтение IP адреса сервера
\item Подключение к серверу
\item Создание потока
\item Чтение данных от сервера и реакция
\item Отправка команд серверу
\item Поток для получения данных от сервера
\end{itemize}


\section{Тестирование}

\begin{figure}[H]
	\begin{center}
		\includegraphics[scale=0.7]{1}
		\caption{Просмотр списка тем} 
		\label{pic:pic_name} % название для ссылок внутри кода
	\end{center}
\end{figure}

\begin{figure}[H]
	\begin{center}
		\includegraphics[scale=0.7]{2}
		\caption{Просмотр название новостей по теме war} 
		\label{pic:pic_name} % название для ссылок внутри кода
	\end{center}
\end{figure}

\begin{figure}[H]
	\begin{center}
		\includegraphics[scale=0.7]{3}
		\caption{Просмотр текста второй новости} 
		\label{pic:pic_name} % название для ссылок внутри кода
	\end{center}
\end{figure}

\begin{figure}[H]
	\begin{center}
		\includegraphics[scale=0.7]{4}
		\caption{Добавление новой новости} 
		\label{pic:pic_name} % название для ссылок внутри кода
	\end{center}
\end{figure}

\begin{figure}[H]
	\begin{center}
		\includegraphics[scale=0.7]{5}
		\caption{Просмотр добавленной новости на втором клиенте} 
		\label{pic:pic_name} % название для ссылок внутри кода
	\end{center}
\end{figure}

\begin{figure}[H]
	\begin{center}
		\includegraphics[scale=0.7]{6}
		\caption{Выход клиента} 
		\label{pic:pic_name} % название для ссылок внутри кода
	\end{center}
\end{figure}

\begin{figure}[H]
	\begin{center}
		\includegraphics[scale=0.7]{7}
		\caption{Приемер вывода ошибки} 
		\label{pic:pic_name} % название для ссылок внутри кода
	\end{center}
\end{figure}


\section{Выводы}

В ходе работы были изучены принципы программирования сокетов с использованием протокола TCP.\\
Во время выполнения индивидуального задания была реализована клиент-серверная программа системы поиска/публикации новостей, с написанием собственного протокола на основе TCP. 
В результате этого были изучены принципы программирования сокетов TCP. Основной проблемой при реализации приложения на TCP была необходимость контроля длины посылки. Ее решением стало добавление символа окончания посылки. TCP требует установления соединения, поэтому на сервере выделяется поток, в котором происходит прием запросов на соединение от клиентов через выделенный для этого сокет. После подключения очередного клиента порождается отдельный поток, осуществляющий обмен пакетами с этим клиентом через отдельный сокет. 


\end{document}
